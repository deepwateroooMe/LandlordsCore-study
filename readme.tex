% Created 2023-05-20 Sat 22:37
\documentclass[9pt, b5paper]{article}
\usepackage{xeCJK}
\usepackage[T1]{fontenc}
\usepackage{bera}
\usepackage[scaled]{beraserif}
\usepackage[scaled]{berasans}
\usepackage[scaled]{beramono}
\usepackage[cache=false]{minted}
\usepackage{xltxtra}
\usepackage{graphicx}
\usepackage{xcolor}
\usepackage{multirow}
\usepackage{multicol}
\usepackage{float}
\usepackage{textcomp}
\usepackage{algorithm}
\usepackage{algorithmic}
\usepackage{latexsym}
\usepackage{natbib}
\usepackage{geometry}
\geometry{left=1.2cm,right=1.2cm,top=1.5cm,bottom=1.2cm}
\usepackage[xetex,colorlinks=true,CJKbookmarks=true,linkcolor=blue,urlcolor=blue,menucolor=blue]{hyperref}
\newminted{common-lisp}{fontsize=\footnotesize} 
\author{deepwaterooo}
\date{\today}
\title{ET框架小游戏--斗地主源码学习--参考来帮助拖拉机重构游戏}
\hypersetup{
  pdfkeywords={},
  pdfsubject={},
  pdfcreator={Emacs 28.2 (Org mode 8.2.7c)}}
\begin{document}

\maketitle
\tableofcontents


\section{UILobby 匹配按钮的回调}
\label{sec-1}
\subsection{【客户端请求】LandlordsLobbyComponent ==> OnStartMatch()}
\label{sec-1-1}
\begin{itemize}
\item 这个类向服务器发消息前,会先检查用户是否余额不足。
\item 初始化时,因为用户是已经登录时来的,所以会去数据库拿用户的信息
\end{itemize}
\begin{minted}[fontsize=\scriptsize,linenos=false]{csharp}
public class LandlordsLobbyComponent : Component { // 大厅界面组件
    public void Awake() {
        Init();
    }
    // 开始匹配按钮事件
    public async void OnStartMatch() { 
        try {
            // 发送开始匹配消息
            C2G_StartMatch_Req c2G_StartMatch_Req = new C2G_StartMatch_Req();
            G2C_StartMatch_Ack g2C_StartMatch_Ack = await SessionComponent.Instance.Session.Call(c2G_StartMatch_Req) as G2C_StartMatch_Ack; // 这里去看下服务器的处理逻辑
            if (g2C_StartMatch_Ack.Error == ErrorCode.ERR_UserMoneyLessError) {
                Log.Error("余额不足"); // 就是说,当且仅当余额不足的时候才会出这个错误?
                return;
            }
            // 匹配成功了:UI 界面切换,切换到房间界面
            UI room = Game.Scene.GetComponent<UIComponent>().Create(UIType.LandlordsRoom); // 装载新的UI视图
            Game.Scene.GetComponent<UIComponent>().Remove(UIType.LandlordsLobby);          // 卸载旧的UI视图
            // 将房间设为匹配状态
            room.GetComponent<LandlordsRoomComponent>().Matching = true;
        }
        catch (Exception e) {
            Log.Error(e.ToStr());
        }
    }
\end{minted}
\subsection{【服务端】 C2G\_StartMatch\_ReqHandler:【网关服】处理来自客户端的匹配请求}
\label{sec-1-2}
\begin{minted}[fontsize=\scriptsize,linenos=false]{csharp}
[MessageHandler(AppType.Gate)] // 网关服:处理客户端 StartMatch 请求消息
public class C2G_StartMatch_ReqHandler : AMRpcHandler<C2G_StartMatch_Req, G2C_StartMatch_Ack> {
    protected override async void Run(Session session, C2G_StartMatch_Req message, Action<G2C_StartMatch_Ack> reply) {
        G2C_StartMatch_Ack response = new G2C_StartMatch_Ack();
        try {
            if (!GateHelper.SignSession(session)) { // 验证Session
                response.Error = ErrorCode.ERR_SignError;
                reply(response);
                return;
            }
            User user = session.GetComponent<SessionUserComponent>().User;
            // 验证玩家是否符合进入房间要求,默认为100底分局
            RoomConfig roomConfig = RoomHelper.GetConfig(RoomLevel.Lv100);// 有不同标准的游戏房间
            UserInfo userInfo = await Game.Scene.GetComponent<DBProxyComponent>().Query<UserInfo>(user.UserID, false); // 跑数据库里去拿,这个玩家的现金验证是否合格
            if (userInfo.Money < roomConfig.MinThreshold) {
                response.Error = ErrorCode.ERR_UserMoneyLessError; // 玩家钱不够,不能玩
                reply(response);
                return;
            }
// 这里先发送响应,让客户端收到后切换房间界面,否则可能会出现重连消息在切换到房间界面之前发送导致重连异常【这个应该是,别人的源标注了】
// 这里的顺序就显得关键:因为只有网关服向客户端返回服务器的匹配响应【并不一定说已经匹配完成,但告诉客户端服务器在着手处理这个工作。。。】,客户端才能创建房间UI 控件
            reply(response); 
            // 向匹配服务器发送匹配请求
            StartConfigComponent config = Game.Scene.GetComponent<StartConfigComponent>();
            IPEndPoint matchIPEndPoint = config.MatchConfig.GetComponent<InnerConfig>().IPEndPoint; // 匹配服务器的远程IP 地址
            Session matchSession = Game.Scene.GetComponent<NetInnerComponent>().Get(matchIPEndPoint); // 拿到与这个匹配服务器通信的会话框实例
            M2G_PlayerEnterMatch_Ack m2G_PlayerEnterMatch_Ack = await matchSession.Call(new G2M_PlayerEnterMatch_Req() { // 发消息代为客户端申请:申请匹配游戏
                    PlayerID = user.InstanceId,
                        UserID = user.UserID,
                        SessionID = session.InstanceId,
                        }) as M2G_PlayerEnterMatch_Ack;
            user.IsMatching = true;
        }
        catch (Exception e) {
            ReplyError(response, e, reply);
        }
    }
}
\end{minted}
\subsection{MatchComponent:【匹配功能?】组件,匹配逻辑在MatchComponentSystem扩展. 这里是处理匹配逻辑的组件:它就需要【申请匹配者人】+【匹配玩家所到的游戏房间】两大部分}
\label{sec-1-3}
\begin{itemize}
\item \textbf{【MatchRoomComponent】} :被匹配到的房间,成为这个组件的另一个组成部分。
\item \textbf{【Matcher被匹配者】} :是这个匹配功能的一大内在版块
\begin{minted}[fontsize=\scriptsize,linenos=false]{csharp}
// 匹配组件,匹配逻辑在MatchComponentSystem扩展. 这里是处理匹配的组件,与 Matcher 被匹配者相区分开来
public class MatchComponent : Component {
    // 游戏中匹配对象列表:值是 roomId
    public readonly Dictionary<long, long> Playing = new Dictionary<long, long>();
    // 匹配成功队列
    public readonly Queue<Matcher> MatchSuccessQueue = new Queue<Matcher>();
    // 创建房间消息加锁,避免因为延迟重复发多次创建房间消息
    public bool CreateRoomLock { get; set; }
}
\end{minted}
\end{itemize}
\subsection{MatchComponentSystem: Update() 更新系。后面跟几个组件及个体类}
\label{sec-1-4}
\begin{itemize}
\item 与服务器的交互再涉及两个类:创建新房间组件,与申请匹配成功的玩家进入房间组件
\begin{minted}[fontsize=\scriptsize,linenos=false]{csharp}
public static class MatchComponentSystem {
    public static void Update(this MatchComponent self) {
        while (true) {
            MatcherComponent matcherComponent = Game.Scene.GetComponent<MatcherComponent>();// 玩家管理组件
            Queue<Matcher> matchers = new Queue<Matcher>(matcherComponent.GetAll());        // 玩家们
            MatchRoomComponent roomManager = Game.Scene.GetComponent<MatchRoomComponent>(); // 游戏房间
            Room room = roomManager.GetReadyRoom(); // 返回的是:人员不满 < 3 个的一个房间      // 房间个体
            if (matchers.Count == 0) 
                // 当没有匹配玩家时直接结束
                break;
            if (room == null && matchers.Count >= 3) // 分配一个空房间
                // 当还有一桌匹配玩家且没有可加入房间时使用空房间
                room = roomManager.GetIdleRoom();
            if (room != null) { // 只要房间不为空,就被强按到这个房间里了,没有任何其它逻辑考量
                // 当有准备状态房间且房间还有空位时匹配玩家直接加入填补空位
                while (matchers.Count > 0 && room.Count < 3) // 是个循环:可以匹配好几个玩家,到好几个有空位的游戏房间 
                    self.JoinRoom(room, matcherComponent.Remove(matchers.Dequeue().UserID));
            }
            else if (matchers.Count >= 3) {
                // 当还有一桌匹配玩家且没有空房间时创建新房间
                self.CreateRoom();
                break;
            } else break;
            // 移除匹配成功玩家
            while (self.MatchSuccessQueue.Count > 0) 
                matcherComponent.Remove(self.MatchSuccessQueue.Dequeue().UserID);
        }
    }
    // 创建房间
    public static async void CreateRoom(this MatchComponent self) {
        if (self.CreateRoomLock) 
            return;
        // 消息加锁,避免因为延迟重复发多次创建消息
        self.CreateRoomLock = true;
        // 发送创建房间消息:这里几个相关组件,可能重构的时候,也会被 ET7 重构去掉,所以没有看。重点看:【大型网络游戏中需要与服务器交互的部分】
        IPEndPoint mapIPEndPoint = Game.Scene.GetComponent<AllotMapComponent>().GetAddress().GetComponent<InnerConfig>().IPEndPoint;
        Session mapSession = Game.Scene.GetComponent<NetInnerComponent>().Get(mapIPEndPoint);
        MP2MH_CreateRoom_Ack createRoomRE = await mapSession.Call(new MH2MP_CreateRoom_Req()) as MP2MH_CreateRoom_Ack; // <<<<<<<<<<<<<<<<<<<< 
        Room room = ComponentFactory.CreateWithId<Room>(createRoomRE.RoomID);
        Game.Scene.GetComponent<MatchRoomComponent>().Add(room);
        // 解锁
        self.CreateRoomLock = false;
    }
    // 加入房间:逻辑极简单,就只要钱够就可以了。多出了房间服务器【任何时候,活宝妹就是一定要嫁给亲爱的表哥!!!】
    public static async void JoinRoom(this MatchComponent self, Room room, Matcher matcher) {
        // 玩家加入房间,移除匹配队列
        self.Playing[matcher.UserID] = room.Id;
        self.MatchSuccessQueue.Enqueue(matcher);
        // 向房间服务器发送玩家进入请求
        ActorMessageSender actorProxy = Game.Scene.GetComponent<ActorMessageSenderComponent>().Get(room.Id);
        IResponse response = await actorProxy.Call(new Actor_PlayerEnterRoom_Req() {
                PlayerID = matcher.PlayerID,
                    UserID = matcher.UserID,
                    SessionID = matcher.GateSessionID
                    });
        Actor_PlayerEnterRoom_Ack actor_PlayerEnterRoom_Ack = response as Actor_PlayerEnterRoom_Ack;
        Gamer gamer = GamerFactory.Create(matcher.PlayerID, matcher.UserID, actor_PlayerEnterRoom_Ack.GamerID);
        room.Add(gamer);
        // 向玩家发送匹配成功消息
        ActorMessageSenderComponent actorProxyComponent = Game.Scene.GetComponent<ActorMessageSenderComponent>();
        ActorMessageSender gamerActorProxy = actorProxyComponent.Get(gamer.PlayerID);
        gamerActorProxy.Send(new Actor_MatchSucess_Ntt() { GamerID = gamer.Id });
    }
}
\end{minted}
\end{itemize}
\subsubsection{MatchRoomComponent: 游戏房间组件,分玩家满,等更多的玩家,和空房间等几种情况}
\label{sec-1-4-1}
\begin{minted}[fontsize=\scriptsize,linenos=false]{csharp}
// 匹配房间管理组件,逻辑在MatchRoomComponentSystem扩展
public class MatchRoomComponent : Component {
    // 所有房间列表
    public readonly Dictionary<long, Room> rooms = new Dictionary<long, Room>();
    // 游戏中房间列表
    public readonly Dictionary<long, Room> gameRooms = new Dictionary<long, Room>();
    // 等待中房间列表
    public readonly Dictionary<long, Room> readyRooms = new Dictionary<long, Room>();
    // 空闲房间列表
    public readonly Queue<Room> idleRooms = new Queue<Room>();
    // 房间总数
    public int TotalCount { get { return this.rooms.Count; } }
    // 游戏中房间数
    public int GameRoomCount { get { return gameRooms.Count; } }
    // 等待中房间数: 只要人数不够的房间,都算等待中。。。。。
    public int ReadyRoomCount { get { return readyRooms.Where(p => p.Value.Count < 3).Count(); } }
    // 空闲房间数
    public int IdleRoomCount { get { return idleRooms.Count; } }
    public override void Dispose() {
        if (this.IsDisposed) 
            return;
        base.Dispose();
        foreach (var room in this.rooms.Values) {
            room.Dispose();
        }
    }
}
\end{minted}
\subsubsection{Room | RoomState:}
\label{sec-1-4-2}
\begin{itemize}
\item 后面,还有个 RoomComponent 管理者类。下一节
\begin{minted}[fontsize=\scriptsize,linenos=false]{csharp}
// 房间状态
public enum RoomState : byte {
    Idle,       
    Ready,      
    Game        
}
public sealed class Room : Entity { // 房间对象
    public readonly Dictionary<long, int> seats = new Dictionary<long, int>();
    public readonly Gamer[] gamers = new Gamer[3];
    // 房间状态
    public RoomState State { get; set; } = RoomState.Idle;
    // 房间玩家数量
    public int Count { get { return seats.Values.Count; } }
    public override void Dispose() 
        if (this.IsDisposed) {
            return;
        base.Dispose();
        seats.Clear();
        for (int i = 0; i < gamers.Length; i++) 
            if (gamers[i] != null) {
                gamers[i].Dispose();
                gamers[i] = null;
            }
        State = RoomState.Idle;
    }
}
\end{minted}
\end{itemize}
\subsubsection{MatcherComponent: 匹配申请者、被匹配者,的管理类组件。管理者类,就管理了所有发出过这个申请的申请者}
\label{sec-1-4-3}
\begin{minted}[fontsize=\scriptsize,linenos=false]{csharp}
// 匹配对象管理组件
public class MatcherComponent : Component {
    private readonly Dictionary<long, Matcher> matchers = new Dictionary<long, Matcher>();
    // 匹配对象数量
    public int Count { get { return matchers.Count; } }
    // 添加匹配对象
    public void Add(Matcher matcher) {
        this.matchers.Add(matcher.UserID, matcher);
    }
    // 获取匹配对象
    public Matcher Get(long id) {
        this.matchers.TryGetValue(id, out Matcher matcher);
        return matcher;
    }
    // 获取所有匹配对象
    public Matcher[] GetAll() {
        return this.matchers.Values.ToArray();
    }
    // 移除匹配对象并返回
    public Matcher Remove(long id) {
        Matcher matcher = Get(id);
        this.matchers.Remove(id);
        return matcher;
    }
    public override void Dispose() {
        if (this.IsDisposed) 
            return;
        base.Dispose();
        foreach (var matcher in this.matchers.Values) {
            matcher.Dispose();
        }
    }
}
\end{minted}
\subsubsection{Matcher: 匹配申请者,被匹配者组件。是指具体的一个个的申请者}
\label{sec-1-4-4}
\begin{itemize}
\item 它像是个自觉醒组件。同一个文件里也添加了 Awake()
\begin{minted}[fontsize=\scriptsize,linenos=false]{csharp}
// 匹配对象: 匹配的玩家系统
public sealed class Matcher : Entity {
    // 用户ID(唯一)
    public long UserID { get; private set; }
    // 玩家GateActorID
    public long PlayerID { get; set; }
    // 客户端与网关服务器的SessionID
    public long GateSessionID { get; set; }
    public void Awake(long id) {
        this.UserID = id;
    }
    public override void Dispose() {
        if(this.IsDisposed) return; 
        base.Dispose();
        this.UserID = 0;
        this.PlayerID = 0;
        this.GateSessionID = 0;
    }
}
\end{minted}
\end{itemize}
\subsection{【服务端】MH2MP\_CreateRoom\_ReqHandler:【地图服】会创建新的游戏房间}
\label{sec-1-5}
\begin{itemize}
\item 工厂化生产了一个房间。并为房间添加了几个管理者类组件:DeckComponent, DeskCardsCacheComponent, OrderControllerComponent, GameControllerComponent,
\item 为游戏房间添加了邮箱组件,方便游戏房间里聊天,“再不出牌我就要打 120 了呀。。活宝妹就是一定要嫁给亲爱的表哥!!!”【活宝妹就是一定要嫁给亲爱的表哥!!!】
\item 把当前刚生产的房间加入管理者的统管范围。RoomComponent
\item 这里只是大致了解,游戏客户端与服务端的交互设计,游戏里元件组件的拆分,里面的连接逻辑,元件组件间的交互逻辑还没有细看。有必要时会细看。
\begin{minted}[fontsize=\scriptsize,linenos=false]{csharp}
[MessageHandler(AppType.Map)]
public class MH2MP_CreateRoom_ReqHandler : AMRpcHandler<MH2MP_CreateRoom_Req, MP2MH_CreateRoom_Ack> {
    protected override async void Run(Session session, MH2MP_CreateRoom_Req message, Action<MP2MH_CreateRoom_Ack> reply) {
        MP2MH_CreateRoom_Ack response = new MP2MH_CreateRoom_Ack();
        try {
            // 创建房间
            Room room = ComponentFactory.Create<Room>(); // 工厂化生产一个房间
            room.AddComponent<DeckComponent>();
            room.AddComponent<DeskCardsCacheComponent>();
            room.AddComponent<OrderControllerComponent>();
            room.AddComponent<GameControllerComponent, RoomConfig>(RoomHelper.GetConfig(RoomLevel.Lv100));
            await room.AddComponent<MailBoxComponent>().AddLocation();// 去查看一下:是否是为了方便游戏房间里聊天?
            Game.Scene.GetComponent<RoomComponent>().Add(room);
            Log.Info($"创建房间{room.InstanceId}");
            response.RoomID = room.InstanceId;
            reply(response);
        }
        catch (Exception e) {
            ReplyError(response, e, reply);
        }
    }
}
\end{minted}
\end{itemize}
\subsubsection{DeckComponent: 牌库组件}
\label{sec-1-5-1}
\begin{minted}[fontsize=\scriptsize,linenos=false]{csharp}
public class DeckComponent : Component { // 牌库组件
    // 牌库中的牌
    public readonly List<Card> library = new List<Card>();
    // 牌库中的总牌数
    public int CardsCount { get { return this.library.Count; } }
    public override void Dispose() {
        if (this.IsDisposed) 
            return;
        base.Dispose();
        library.Clear();
    }
}
\end{minted}
\subsubsection{DeskCardsCacheComponent: 上面一个组件可能不够用,不得不加几个组件来组合}
\label{sec-1-5-2}
\begin{minted}[fontsize=\scriptsize,linenos=false]{csharp}
public class DeskCardsCacheComponent : Component {
    // 牌桌上的牌
    public readonly List<Card> library = new List<Card>();
    // 地主牌
    public readonly List<Card> LordCards = new List<Card>();
    // 牌桌上的总牌数
    public int CardsCount { get { return this.library.Count; } }
    // 当前最大牌型: 这里为什么要纪录当前最大牌型?哪家的?读源码来搞明白
    public CardsType Rule { get; set; }
    // 牌桌上最小的牌
    public int MinWeight { get { return (int)this.library[0].CardWeight; } }
    public override void Dispose() {
        if (this.IsDisposed) 
            return;
        base.Dispose();
        library.Clear();
        LordCards.Clear();
        Rule = CardsType.None;
    }
}
\end{minted}
\subsubsection{OrderControllerComponent: 玩家出牌顺序什么之类的游戏逻辑的管理}
\label{sec-1-5-3}
\begin{minted}[fontsize=\scriptsize,linenos=false]{csharp}
// 这些都算是:游戏逻辑控制的组件化拆分。以前自己的游戏可能是一个巨大无比的控制器文件,这里折分成了狠多个小组件控制
public class OrderControllerComponent : Component {
    // 先手玩家
    public KeyValuePair<long, bool> FirstAuthority { get; set; }
    // 玩家抢地主状态
    public Dictionary<long, bool> GamerLandlordState = new Dictionary<long, bool>();
    // 本轮最大牌型玩家
    public long Biggest { get; set; }
    // 当前出牌玩家
    public long CurrentAuthority { get; set; }
    // 当前抢地主玩家
    public int SelectLordIndex { get; set; }

    public override void Dispose() {
        if (this.IsDisposed) 
            return;
        base.Dispose();
        this.GamerLandlordState.Clear();
        this.Biggest = 0;
        this.CurrentAuthority = 0;
        this.SelectLordIndex = 0;
    }
}
\end{minted}
\subsubsection{GameControllerComponent: 游戏控制类}
\label{sec-1-5-4}
\begin{minted}[fontsize=\scriptsize,linenos=false]{csharp}
// 感觉个类,更多的是【一座桥】:把游戏的这个单位级件,全连接起来
public class GameControllerComponent : Component {
    // 房间配置
    public RoomConfig Config { get; set; }
    // 底分: 这里呈现出与房间的这些设置不一致的状态。是说,三个玩家,可以在既定房间的基础上提升玩乐标准?
    public long BasePointPerMatch { get; set; }
    // 全场倍率
    public int Multiples { get; set; }
    // 最低入场门槛
    public long MinThreshold { get; set; }

    public override void Dispose() {
        if (this.IsDisposed) return;
        base.Dispose();
        this.BasePointPerMatch = 0;
        this.Multiples = 0;
        this.MinThreshold = 0;
    }
}
\end{minted}
\subsubsection{RoomComponent: 房间管理组件}
\label{sec-1-5-5}
\begin{itemize}
\item ET 框架源码读多也,也该明白,所有的 Component 组件,全都是管理者组件。
\begin{minted}[fontsize=\scriptsize,linenos=false]{csharp}
// 房间管理组件
public class RoomComponent : Component {
    private readonly Dictionary<long, Room> rooms = new Dictionary<long, Room>();
    // 添加房间
    public void Add(Room room) {
        this.rooms.Add(room.InstanceId, room);
    }
    // 获取房间
    public Room Get(long id) {
        Room room;
        this.rooms.TryGetValue(id, out room);
        return room;
    }
    // 移除房间并返回
    public Room Remove(long id) {
        Room room = Get(id);
        this.rooms.Remove(id);
        return room;
    }
    public override void Dispose() {
        if (this.IsDisposed) return;
        base.Dispose();
        foreach (var room in this.rooms.Values) {
            room.Dispose();
        }
    }
}
\end{minted}
\end{itemize}
\subsubsection{RoomConfig: 房间配置,房间的基本参数,什么的}
\label{sec-1-5-6}
\begin{minted}[fontsize=\scriptsize,linenos=false]{csharp}
// 房间配置
public struct RoomConfig {
    // 房间初始倍率
    public int Multiples { get; set; }
    // 房间底分
    public long BasePointPerMatch { get; set; }
    // 房间最低门槛
    public long MinThreshold { get; set; }
}
\end{minted}

\subsection{Actor\_PlayerEnterRoom\_ReqHandler: 玩家进入游戏房间}
\label{sec-1-6}
\begin{itemize}
\item 为玩家添加邮箱,方便玩家收发消息。那前面,为什么房间也要添加邮箱?集中消息?可是每个玩家看见的都是自己的往返消息,集中消息给谁看?
\item 广播:新玩家进场
\item 通过代理发送:【游戏开始】的消息?不知道这个消息是怎么处理的。逻辑不通,每个玩家都发,谁说了算,得查逻辑
\begin{minted}[fontsize=\scriptsize,linenos=false]{csharp}
[ActorMessageHandler(AppType.Map)]
public class Actor_PlayerEnterRoom_ReqHandler : AMActorRpcHandler<Room, Actor_PlayerEnterRoom_Req, Actor_PlayerEnterRoom_Ack> {
    protected override async Task Run(Room room, Actor_PlayerEnterRoom_Req message, Action<Actor_PlayerEnterRoom_Ack> reply) {
        Actor_PlayerEnterRoom_Ack response = new Actor_PlayerEnterRoom_Ack();
        try {
            Gamer gamer = room.Get(message.UserID);
            if (gamer == null) { // 当前玩家,在这个被分配的房间里,还没被初始化
                // 创建房间玩家对象
                gamer = GamerFactory.Create(message.PlayerID, message.UserID);
                await gamer.AddComponent<MailBoxComponent>().AddLocation(); // 只有给玩家挂上这个组件,并向中央邮件注册登记地址,接下来的游戏它才可以收发消息,出牌什么的
                gamer.AddComponent<UnitGateComponent, long>(message.SessionID);
                // 加入到房间
                room.Add(gamer); // 这里就又多一步逻辑处理:这里当服务器匹配成功一个玩家,就去做相应的客户端视图层相应的变动调动
                Actor_GamerEnterRoom_Ntt broadcastMessage = new Actor_GamerEnterRoom_Ntt();
                foreach (Gamer _gamer in room.GetAll()) {
                    if (_gamer == null) {
                        // 添加空位: 添加所有的,当前这个消息的接受者
                        broadcastMessage.Gamers.Add(new GamerInfo());
                        continue;
                    }
                    // 添加玩家信息
                    GamerInfo info = new GamerInfo() { UserID = _gamer.UserID, IsReady = _gamer.IsReady };
                    broadcastMessage.Gamers.Add(info);
                }
                // 广播消息:给房间内的所有玩家,新人驾到,请多关照
                room.Broadcast(broadcastMessage);
                Log.Info($"玩家{message.UserID}进入房间");
            } else { // 【任何时候,活宝妹就是一定要、一定会嫁给偶亲爱的表哥!!!】
                // 玩家重连
                gamer.isOffline = false;
                gamer.PlayerID = message.PlayerID;
                gamer.GetComponent<UnitGateComponent>().GateSessionActorId = message.SessionID;
                // 玩家重连,移除托管组件
                gamer.RemoveComponent<TrusteeshipComponent>(); // 这个好像是使玩家可以自动机器人帮出牌的
                Actor_GamerEnterRoom_Ntt broadcastMessage = new Actor_GamerEnterRoom_Ntt();
                foreach (Gamer _gamer in room.GetAll()) {
                    if (_gamer == null) {
                        // 添加空位
                        broadcastMessage.Gamers.Add(default(GamerInfo));
                        continue;
                    }
                    // 添加玩家信息
                    GamerInfo info = new GamerInfo() { UserID = _gamer.UserID, IsReady = _gamer.IsReady };
                    broadcastMessage.Gamers.Add(info);
                }
                // 发送房间玩家信息
                ActorMessageSender actorProxy = gamer.GetComponent<UnitGateComponent>().GetActorMessageSender();
                actorProxy.Send(broadcastMessage);
                // 这部分:看看清楚 
                List<GamerCardNum> gamersCardNum = new List<GamerCardNum>();
                List<GamerState> gamersState = new List<GamerState>();
                GameControllerComponent gameController = room.GetComponent<GameControllerComponent>();
                OrderControllerComponent orderController = room.GetComponent<OrderControllerComponent>();
                DeskCardsCacheComponent deskCardsCache = room.GetComponent<DeskCardsCacheComponent>();
                foreach (Gamer _gamer in room.GetAll()) {
                    HandCardsComponent handCards = _gamer.GetComponent<HandCardsComponent>(); // 游戏开始里,Actor_GameStart_NttHandler 会为玩家添加手牌
                    gamersCardNum.Add(new GamerCardNum() {
                            UserID = _gamer.UserID,
                                Num = _gamer.GetComponent<HandCardsComponent>().GetAll().Length
                                });
                    GamerState gamerState = new GamerState() {
                        UserID = _gamer.UserID,
                        UserIdentity = handCards.AccessIdentity
                    };
                    if (orderController.GamerLandlordState.TryGetValue(_gamer.UserID, out bool state)) {
                        if (state) 
                            gamerState.State = GrabLandlordState.Grab;
                        else 
                            gamerState.State = GrabLandlordState.UnGrab;
                    }
                    gamersState.Add(gamerState);
                }
                // 发送游戏开始消息
                Actor_GameStart_Ntt gameStartNotice = new Actor_GameStart_Ntt(); // 因为这个逻辑比较多,后面的没有再看
                gameStartNotice.HandCards.AddRange(gamer.GetComponent<HandCardsComponent>().GetAll());
                gameStartNotice.GamersCardNum.AddRange(gamersCardNum);
                actorProxy.Send(gameStartNotice);
                Card[] lordCards = null;
                if (gamer.GetComponent<HandCardsComponent>().AccessIdentity == Identity.None) {
                    // 广播先手玩家
                    actorProxy.Send(new Actor_AuthorityGrabLandlord_Ntt() { UserID = orderController.CurrentAuthority });
                } else {
                    if (gamer.UserID == orderController.CurrentAuthority) {
                        // 发送可以出牌消息
                        bool isFirst = gamer.UserID == orderController.Biggest;
                        actorProxy.Send(new Actor_AuthorityPlayCard_Ntt() { UserID = orderController.CurrentAuthority, IsFirst = isFirst });
                    }
                    lordCards = deskCardsCache.LordCards.ToArray();
                }
                // 发送重连数据
                Actor_GamerReconnect_Ntt reconnectNotice = new Actor_GamerReconnect_Ntt() {
                    UserId = orderController.Biggest,
                    Multiples = room.GetComponent<GameControllerComponent>().Multiples
                };
                reconnectNotice.GamersState.AddRange(gamersState);
                reconnectNotice.Cards.AddRange(deskCardsCache.library);
                if (lordCards != null) 
                    reconnectNotice.LordCards.AddRange(lordCards);
                actorProxy.Send(reconnectNotice);
                Log.Info($"玩家{message.UserID}重连");
            }
            response.GamerID = gamer.InstanceId;
            reply(response);
        }
        catch (Exception e) {
            ReplyError(response, e, reply);
        }
    }
}
\end{minted}
\end{itemize}
\subsubsection{UnitGateComponent|UnitGateComponentAwakeSystem}
\label{sec-1-6-1}
\begin{itemize}
\item 有了这个组件,好像是玩家间就可以发消息了?
\begin{minted}[fontsize=\scriptsize,linenos=false]{csharp}
[ObjectSystem]
public class UnitGateComponentAwakeSystem : AwakeSystem<UnitGateComponent, long> {
    public override void Awake(UnitGateComponent self, long a) {
        self.Awake(a);
    }
}
public class UnitGateComponent : Component, ISerializeToEntity {
    public long GateSessionActorId;
    public bool IsDisconnect;
    public void Awake(long gateSessionId) {
        this.GateSessionActorId = gateSessionId;
    }
    public ActorMessageSender GetActorMessageSender() {
        return Game.Scene.GetComponent<ActorMessageSenderComponent>().Get(this.GateSessionActorId);
    }
}
\end{minted}
\end{itemize}
\subsubsection{RoomSystem: 房间内部逻辑生成系,可以添加移除玩家、广播消息等}
\label{sec-1-6-2}
\begin{minted}[fontsize=\scriptsize,linenos=false]{csharp}
public static class RoomSystem {
    // 添加玩家
    public static void Add(this Room self, Gamer gamer) {
        int seatIndex = self.GetEmptySeat();
        // 玩家需要获取一个座位坐下
        if (seatIndex >= 0) {
            self.gamers[seatIndex] = gamer;
            self.seats[gamer.UserID] = seatIndex;
            gamer.RoomID = self.InstanceId;
        }
    }
    // 获取玩家
    public static Gamer Get(this Room self, long id) {
        int seatIndex = self.GetGamerSeat(id);
        if (seatIndex >= 0) 
            return self.gamers[seatIndex];
        return null;
    }
    // 获取所有玩家
    public static Gamer[] GetAll(this Room self) {
        return self.gamers;
    }
    // 获取玩家座位索引
    public static int GetGamerSeat(this Room self, long id) {
        if (self.seats.TryGetValue(id, out int seatIndex)) 
            return seatIndex;
        return -1;
    }
    // 移除玩家并返回
    public static Gamer Remove(this Room self, long id) {
        int seatIndex = self.GetGamerSeat(id);
        if (seatIndex >= 0) {
            Gamer gamer = self.gamers[seatIndex];
            self.gamers[seatIndex] = null;
            self.seats.Remove(id);
            gamer.RoomID = 0;
            return gamer;
        }
        return null;
    }
    // 获取空座位
    // <returns>返回座位索引,没有空座位时返回-1</returns>
    public static int GetEmptySeat(this Room self) {
        for (int i = 0; i < self.gamers.Length; i++) 
            if (self.gamers[i] == null) 
                return i;
        return -1;
    }
    // 广播消息
    public static void Broadcast(this Room self, IActorMessage message) {
        foreach (Gamer gamer in self.gamers) {
            if (gamer == null || gamer.isOffline) 
                continue;
            ActorMessageSender actorProxy = gamer.GetComponent<UnitGateComponent>().GetActorMessageSender();
            actorProxy.Send(message);
        }
    }
}
\end{minted}
\subsubsection{GamerState: 玩家状态消息, id, UserIdentity, 是地主吗?}
\label{sec-1-6-3}
\begin{minted}[fontsize=\scriptsize,linenos=false]{csharp}
message GamerState {
    int64 UserID = 1;
    ETModel.Identity UserIdentity = 2;
	GrabLandlordState State = 3;
}
\end{minted}
\subsubsection{HandCardsComponent: 为进入了(和正在处理进入)房间的玩家,添加手里的牌组件}
\label{sec-1-6-4}
\begin{minted}[fontsize=\scriptsize,linenos=false]{csharp}
public class HandCardsComponent : Component {
    // 所有手牌
    public readonly List<Card> library = new List<Card>();
    // 身份:地主,还是平民老百姓?
    public Identity AccessIdentity { get; set; }
    // 是否托管:自动出牌吗
    public bool IsTrusteeship { get; set; }
    // 手牌数
    public int CardsCount { get { return library.Count; } }
    public override void Dispose() {
        if (this.IsDisposed) return;
        base.Dispose();
        this.library.Clear();
        AccessIdentity = Identity.None;
        IsTrusteeship = false;
    }
}
\end{minted}

\subsection{Actor\_GameStart\_NttHandler: 游戏开始逻辑处理}
\label{sec-1-7}
\begin{itemize}
\item 主要是以牌桌上的:什么地主牌呀,一些清空(选中选中过的牌,就是跳高跳出来准备出的,为什么先前没有处理?),重置,等预处理
\item 牌桌,地主牌
\begin{minted}[fontsize=\scriptsize,linenos=false]{csharp}
[MessageHandler]
public class Actor_GameStart_NttHandler : AMHandler<Actor_GameStart_Ntt> {
    protected override void Run(ETModel.Session session, Actor_GameStart_Ntt message) {
        UI uiRoom = Game.Scene.GetComponent<UIComponent>().Get(UIType.LandlordsRoom);
        GamerComponent gamerComponent = uiRoom.GetComponent<GamerComponent>();
        // 初始化玩家UI
        foreach (GamerCardNum gamerCardNum in message.GamersCardNum) {
            Gamer gamer = uiRoom.GetComponent<GamerComponent>().Get(gamerCardNum.UserID);
            GamerUIComponent gamerUI = gamer.GetComponent<GamerUIComponent>();
            gamerUI.GameStart();
            HandCardsComponent handCards = gamer.GetComponent<HandCardsComponent>();
            if (handCards != null) 
                handCards.Reset();
            else 
                handCards = gamer.AddComponent<HandCardsComponent, GameObject>(gamerUI.Panel);
            handCards.Appear();
            if (gamer.UserID == gamerComponent.LocalGamer.UserID) 
                // 本地玩家添加手牌
                handCards.AddCards(message.HandCards);
            else 
                // 设置其他玩家手牌数
                handCards.SetHandCardsNum(gamerCardNum.Num);
        }
        // 显示牌桌UI
        GameObject desk = uiRoom.GameObject.Get<GameObject>("Desk");
        desk.SetActive(true);
        GameObject lordPokers = desk.Get<GameObject>("LordPokers");
        // 重置地主牌
        Sprite lordSprite = CardHelper.GetCardSprite("None");
        for (int i = 0; i < lordPokers.transform.childCount; i++) 
            lordPokers.transform.GetChild(i).GetComponent<Image>().sprite = lordSprite;
        LandlordsRoomComponent uiRoomComponent = uiRoom.GetComponent<LandlordsRoomComponent>();
        // 清空选中牌
        uiRoomComponent.Interaction.Clear();
        // 设置初始倍率
        uiRoomComponent.SetMultiples(1);
    }
}
\end{minted}
\end{itemize}
\subsubsection{GamerUIComponent: 每个玩家身上所背的UI 小面板,用来显示这个玩家的相关信息}
\label{sec-1-7-1}
\begin{itemize}
\item 包括UI 面板,必要的用户信息,玩家昵称
\item 功能包括:抢不抢地主什么的
\item LandlordsInteractionComponent: 它说,这个组件更像是出牌互动功能模块?
\begin{minted}[fontsize=\scriptsize,linenos=false]{csharp}
// 每个玩家的UI组件
public class GamerUIComponent : Component {
    // UI面板
    public GameObject Panel { get; private set; }
    // 玩家昵称
    public string NickName { get { return name.text; } }
    private Image headPhoto;
    private Text prompt;
    private Text name;
    private Text money;
    public void Start() {
        if (this.GetParent<Gamer>().IsReady) 
            SetReady();
    }
    // 重置面板
    public void ResetPanel() {
        ResetPrompt();
        this.headPhoto.gameObject.SetActive(false);
        this.name.text = "空位";
        this.money.text = "";
        this.Panel = null;
        this.prompt = null;
        this.name = null;
        this.money = null;
        this.headPhoto = null;
    }
    // 设置面板
    public void SetPanel(GameObject panel) {
        this.Panel = panel;
        // 绑定关联
        this.prompt = this.Panel.Get<GameObject>("Prompt").GetComponent<Text>();
        this.name = this.Panel.Get<GameObject>("Name").GetComponent<Text>();
        this.money = this.Panel.Get<GameObject>("Money").GetComponent<Text>();
        this.headPhoto = this.Panel.Get<GameObject>("HeadPhoto").GetComponent<Image>();
        UpdatePanel();
    }
    // 更新面板
    public void UpdatePanel() {
        if (this.Panel != null) {
            SetUserInfo();
            headPhoto.gameObject.SetActive(false);
        }
    }
    // 设置玩家身份
    public void SetIdentity(Identity identity) {
        if (identity == Identity.None)
            return;
        string spriteName = $"Identity_{Enum.GetName(typeof(Identity), identity)}";
        Sprite headSprite = CardHelper.GetCardSprite(spriteName);
        headPhoto.sprite = headSprite;
        headPhoto.gameObject.SetActive(true);
    }
    // 玩家准备
    public void SetReady() {// 去搜一下:为什么会需要这个准备按钮?
        prompt.text = "准备!";
    }
    // 出牌错误
    public void SetPlayCardsError() {
        prompt.text = "您出的牌不符合规则!";
    }
    // 玩家不出
    public void SetDiscard() {
        prompt.text = "不出";
    }
    // 玩家抢地主:抢不抢地主
    public void SetGrab(GrabLandlordState state) {
        switch (state) {
            case GrabLandlordState.Not:
                break;
            case GrabLandlordState.Grab:
                prompt.text = "抢地主";
                break;
            case GrabLandlordState.UnGrab:
                prompt.text = "不抢";
                break;
        }
    }
    // 重置提示
    public void ResetPrompt() {
        prompt.text = "";
    }
    // 游戏开始
    public void GameStart() {
        ResetPrompt();
    }
    // 设置用户信息
    private async void SetUserInfo() {
        G2C_GetUserInfo_Ack g2C_GetUserInfo_Ack = await SessionComponent.Instance.Session.Call(new C2G_GetUserInfo_Req() { UserID = this.GetParent<Gamer>().UserID }) as G2C_GetUserInfo_Ack;
        if (this.Panel != null) {
            name.text = g2C_GetUserInfo_Ack.NickName;
            money.text = g2C_GetUserInfo_Ack.Money.ToString();
        }
    }
    public override void Dispose() {
        if (this.IsDisposed) 
            return;
        base.Dispose();
        // 重置玩家UI
        ResetPanel();
    }
}
\end{minted}
\end{itemize}

\subsection{C2G\_GetUserInfo\_ReqHandler:【网关服】处理用户信息查询请求:它就去拿数据库相关组件。因为 ET7 重构了数据库模块,这里这略过。}
\label{sec-1-8}
\begin{minted}[fontsize=\scriptsize,linenos=false]{csharp}
[MessageHandler(AppType.Gate)]
public class C2G_GetUserInfo_ReqHandler : AMRpcHandler<C2G_GetUserInfo_Req, G2C_GetUserInfo_Ack> {
    protected override async void Run(Session session, C2G_GetUserInfo_Req message, Action<G2C_GetUserInfo_Ack> reply) {
        G2C_GetUserInfo_Ack response = new G2C_GetUserInfo_Ack();
        try {
            // 验证Session
            if (!GateHelper.SignSession(session)) {
                response.Error = ErrorCode.ERR_SignError;
                reply(response);
                return;
            }
            // 查询用户信息
            DBProxyComponent dbProxyComponent = Game.Scene.GetComponent<DBProxyComponent>();
            UserInfo userInfo = await dbProxyComponent.Query<UserInfo>(message.UserID, false);
            response.NickName = userInfo.NickName;
            response.Wins = userInfo.Wins;
            response.Loses = userInfo.Loses;
            response.Money = userInfo.Money;
            reply(response);
        }
        catch (Exception e) {
            ReplyError(response, e, reply);
        }
    }
}
\end{minted}


\section{准备按钮: Prompt-text, 可是它是可点击的按钮 promptButton}
\label{sec-2}
\subsection{promptButton: OnPrompt() 回调函数}
\label{sec-2-1}
\begin{itemize}
\item 从游戏界面上看,这个回调后,本地玩家的手牌显示出来了。地主的三张牌画出背面,不能看
\item 并在本地玩家手牌显示出来后,玩家看过自己的手牌,接下来可以决定:是否抢地主
\begin{minted}[fontsize=\scriptsize,linenos=false]{csharp}
private async void OnPrompt() { // 提示
    Actor_GamerPrompt_Req request = new Actor_GamerPrompt_Req();
    Actor_GamerPrompt_Ack response = await SessionComponent.Instance.Session.Call(request) as Actor_GamerPrompt_Ack;
    GamerComponent gamerComponent = this.GetParent<UI>().GetParent<UI>().GetComponent<GamerComponent>();
    HandCardsComponent handCards = gamerComponent.LocalGamer.GetComponent<HandCardsComponent>();
    // 清空当前选中
    while (currentSelectCards.Count > 0) {
        Card selectCard = currentSelectCards[currentSelectCards.Count - 1];
        handCards.GetSprite(selectCard).GetComponent<HandCardSprite>().OnClick(null);
    }
    // 自动选中提示出牌:这应该是出牌辅助,
    if (response.Cards != null) 
        foreach (Card card in response.Cards) {
            handCards.GetSprite(card).GetComponent<HandCardSprite>().OnClick(null);
        }
}
\end{minted}
\end{itemize}
\subsection{Actor\_GamerPrompt\_ReqHandler:【地图服】说,你不是假人,那我就给你分牌,给你手里的牌排好序,然后下发给你【客户端】}
\label{sec-2-2}
\begin{minted}[fontsize=\scriptsize,linenos=false]{csharp}
[ActorMessageHandler(AppType.Map)]
public class Actor_GamerPrompt_ReqHandler : AMActorRpcHandler<Gamer, Actor_GamerPrompt_Req, Actor_GamerPrompt_Ack> {
    protected override async Task Run(Gamer gamer, Actor_GamerPrompt_Req message, Action<Actor_GamerPrompt_Ack> reply) {
        Actor_GamerPrompt_Ack response = new Actor_GamerPrompt_Ack();
        try {
            Room room = Game.Scene.GetComponent<RoomComponent>().Get(gamer.RoomID);
            OrderControllerComponent orderController = room.GetComponent<OrderControllerComponent>();
            DeskCardsCacheComponent deskCardsCache = room.GetComponent<DeskCardsCacheComponent>();
            List<Card> handCards = new List<Card>(gamer.GetComponent<HandCardsComponent>().GetAll());
            CardsHelper.SortCards(handCards);
            if (gamer.UserID == orderController.Biggest) {// 这个牌序,大小值,还没有看
                response.Cards.AddRange(handCards.Where(card => card.CardWeight == handCards[handCards.Count - 1].CardWeight).ToArray());
            }
            else {
                List<IList<Card>> result = await CardsHelper.GetPrompt(handCards, deskCardsCache, deskCardsCache.Rule);
                if (result.Count > 0) 
                    response.Cards.AddRange(result[RandomHelper.RandomNumber(0, result.Count)]);
            }
            reply(response);
        }
        catch (Exception e) {
            ReplyError(response, e, reply);
        }
    }
}
\end{minted}
\subsection{HandCardsComponentSystem: 处理玩家手牌的一些相关信息}
\label{sec-2-3}
\begin{minted}[fontsize=\scriptsize,linenos=false]{csharp}
public static class HandCardsComponentSystem {
    // 获取所有手牌
    public static Card[] GetAll(this HandCardsComponent self) {
        return self.library.ToArray();//self.library:readonly 是谁什么时候给它赋值的,就是它的手牌,一个玩家的手牌是怎么拿到的?
    }
    // 向牌库中添加牌
    public static void AddCard(this HandCardsComponent self, Card card) {
        self.library.Add(card);// 这里一张张加进来的,从哪里什么时候加进来的?游戏开始时,会添加本地玩家后里的牌
    }
    // 出牌
    public static void PopCard(this HandCardsComponent self, Card card) {
        self.library.Remove(card);
    }
    // 手牌排序
    public static void Sort(this HandCardsComponent self) {
        CardsHelper.SortCards(self.library);
    }
}
\end{minted}
\subsection{DeskCardsCacheComponentSystem: 发牌,添加牌,获取总值等相关操作,逻辑}
\label{sec-2-4}
\begin{minted}[fontsize=\scriptsize,linenos=false]{csharp}
public static class DeskCardsCacheComponentSystem {
    // 获取总权值
    public static int GetTotalWeight(this DeskCardsCacheComponent self) {
        return CardsHelper.GetWeight(self.library.ToArray(), self.Rule);
    }
    // 获取牌桌所有牌
    public static Card[] GetAll(this DeskCardsCacheComponent self) {
        return self.library.ToArray();
    }
    // 发牌
    public static Card Deal(this DeskCardsCacheComponent self) {
        Card card = self.library[self.CardsCount - 1];
        self.library.Remove(card);
        return card;
    }
    // 向牌库中添加牌
    public static void AddCard(this DeskCardsCacheComponent self, Card card) {
        self.library.Add(card);
    }
    // 清空牌桌
    public static void Clear(this DeskCardsCacheComponent self) {
        DeckComponent deck = self.GetParent<Entity>().GetComponent<DeckComponent>();
        while (self.CardsCount > 0) {
            Card card = self.library[self.CardsCount - 1];
            self.library.Remove(card);
            deck.AddCard(card);
        }
        self.Rule = CardsType.None;
    }
    // 手牌排序
    public static void Sort(this DeskCardsCacheComponent self) {
        CardsHelper.SortCards(self.library);
    }
}
\end{minted}

\section{抢地主按钮:抢与不抢}
\label{sec-3}
\subsection{Actor\_GamerGrabLandlordSelect\_NttHandler:【地图服】处理 抢地主逻辑。}
\label{sec-3-1}
\begin{minted}[fontsize=\scriptsize,linenos=false]{csharp}
[ActorMessageHandler(AppType.Map)]
public class Actor_GamerGrabLandlordSelect_NttHandler : AMActorHandler<Gamer, Actor_GamerGrabLandlordSelect_Ntt> {
    protected override void Run(Gamer gamer, Actor_GamerGrabLandlordSelect_Ntt message) {
        Room room = Game.Scene.GetComponent<RoomComponent>().Get(gamer.RoomID);
        OrderControllerComponent orderController = room.GetComponent<OrderControllerComponent>();
        GameControllerComponent gameController = room.GetComponent<GameControllerComponent>();
        if (orderController.CurrentAuthority == gamer.UserID) {
            // 保存玩家、抢地主意愿
            orderController.GamerLandlordState[gamer.UserID] = message.IsGrab;
            if (message.IsGrab) {
                orderController.Biggest = gamer.UserID;
                gameController.Multiples *= 2; // 只要有人抢,就翻倍?
                room.Broadcast(new Actor_SetMultiples_Ntt() { Multiples = gameController.Multiples }); // 广播翻倍
            }
            // 转发消息
            Actor_GamerGrabLandlordSelect_Ntt transpond = new Actor_GamerGrabLandlordSelect_Ntt();
            transpond.IsGrab = message.IsGrab;
            transpond.UserID = gamer.UserID;
            room.Broadcast(transpond);
            if (orderController.SelectLordIndex >= room.Count) {
                 // * 地主:√ 农民1:× 农民2:×
                 // * 地主:× 农民1:√ 农民2:√
                 // * 地主:√ 农民1:√ 农民2:√ 地主:√
                if (orderController.Biggest == 0) {
                    // 没人抢地主则重新发牌
                    gameController.BackToDeck();
                    gameController.DealCards();
                    // 发送玩家手牌
                    Gamer[] gamers = room.GetAll();
                    List<GamerCardNum> gamersCardNum = new List<GamerCardNum>();
                    Array.ForEach(gamers, _gamer => gamersCardNum.Add(new GamerCardNum() {
                            UserID = _gamer.UserID,
                            Num = _gamer.GetComponent<HandCardsComponent>().GetAll().Length
                            }));
                    Array.ForEach(gamers, _gamer =>
                        {
                        ActorMessageSender actorProxy = _gamer.GetComponent<UnitGateComponent>().GetActorMessageSender();
                        Actor_GameStart_Ntt actorMessage = new Actor_GameStart_Ntt();
                        actorMessage.HandCards.AddRange(_gamer.GetComponent<HandCardsComponent>().GetAll());
                        actorMessage.GamersCardNum.AddRange(gamersCardNum);
                        });
                    // 随机先手玩家
                    gameController.RandomFirstAuthority();
                    return;
                }
                else if ((orderController.SelectLordIndex == room.Count &&
                          ((orderController.Biggest != orderController.FirstAuthority.Key && !orderController.FirstAuthority.Value) ||
                           orderController.Biggest == orderController.FirstAuthority.Key)) ||
                         orderController.SelectLordIndex > room.Count) {
                    gameController.CardsOnTable(orderController.Biggest); // 开始出牌了
                    return;
                }
            }
            // 当所有玩家都抢地主时先手玩家还有一次抢地主的机会
            if (gamer.UserID == orderController.FirstAuthority.Key && message.IsGrab) 
                orderController.FirstAuthority = new KeyValuePair<long, bool>(gamer.UserID, true);
            orderController.Turn(); // 轮家
            orderController.SelectLordIndex++; // 轮家抢地方,就给了本地玩家先抢地主的机会,以及其它二家之后的再一次抢的机会。。
            room.Broadcast(new Actor_AuthorityGrabLandlord_Ntt() { UserID = orderController.CurrentAuthority });// 给本地玩家再抢一次的机会
        }
    }
}
\end{minted}
\subsection{Actor\_AuthorityGrabLandlord\_NttHandler}
\label{sec-3-2}
\begin{minted}[fontsize=\scriptsize,linenos=false]{csharp}
[MessageHandler]
public class Actor_AuthorityGrabLandlord_NttHandler : AMHandler<Actor_AuthorityGrabLandlord_Ntt> {
    protected override void Run(ETModel.Session session, Actor_AuthorityGrabLandlord_Ntt message) {
        UI uiRoom = Game.Scene.GetComponent<UIComponent>().Get(UIType.LandlordsRoom);
        GamerComponent gamerComponent = uiRoom.GetComponent<GamerComponent>();
        if (message.UserID == gamerComponent.LocalGamer.UserID) {
            // 显示抢地主交互
            uiRoom.GetComponent<LandlordsRoomComponent>().Interaction.StartGrab();// 就是两个按钮:抢与不抢,重新激活
        }
    }
}
\end{minted}
\subsection{Actor\_SetLandlord\_NttHandler: 设置地主:亲爱的表哥永远是活宝妹这里最受尊重爱护的亲爱的表哥~~!!任何时候,活宝妹就是一定要、一定会嫁给偶亲爱的表哥!!!}
\label{sec-3-3}

\subsection{出牌:还没看通,不知道这个界面怎么出来的}
\label{sec-3-4}
\section{源码梳理:用作【参考项目】来指导拖拉机项目的重构。}
\label{sec-4}
\begin{itemize}
\item 这个文件,以前不知道总结的是些什么乱七八糟的。现在重点梳理:斗地主的游戏逻辑相关
\item 目的是用作参考,来指导自己【拖拉机游戏】的重构。
\item 所以就按照界面相关的形式,或是几个按钮回调的形式来梳理游戏逻辑的【客户端】请求与【服务端】的处理请求
\item 把这些慢慢看得差不多,就可以试着开始想:拖拉机要怎么才能设计成这川可以与服务器交互的多人网络游戏呢?【任何时候,活宝妹就是一定要、一定会嫁给偶亲爱的表哥!!!】
\item 【任何时候,活宝妹就是一定要、一定会嫁给偶亲爱的表哥!!!爱表哥,爱生活!!!】
\end{itemize}
% Emacs 28.2 (Org mode 8.2.7c)
\end{document}